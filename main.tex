\documentclass[11pt]{article}

% ---------- Page and typography ----------
\usepackage[a4paper,margin=1in]{geometry}
\usepackage{microtype}

% ---------- Math ----------
\usepackage{amsmath,amssymb,amsthm,mathtools,bm}
\usepackage{bbm} % for \mathbbm{1} indicator

% ---------- Lists, figures, links ----------
\usepackage{enumitem}
\usepackage{graphicx}
\usepackage[colorlinks=true,linkcolor=blue,citecolor=blue,urlcolor=blue]{hyperref}

% ---------- Short commands ----------
% Sets and common objects
\newcommand{\R}{\mathbb{R}}
\newcommand{\N}{\mathbb{N}}
\newcommand{\Z}{\mathbb{Z}}
\newcommand{\Q}{\mathbb{Q}}
\newcommand{\C}{\mathbb{C}}

% Probability
\newcommand{\bP}{\mathbb{P}}
\newcommand{\E}{\mathbb{E}}
\newcommand{\Var}{\mathrm{Var}}
\newcommand{\Cov}{\mathrm{Cov}}
\newcommand{\Law}{\mathcal{L}}
\newcommand{\1}{\mathbbm{1}}

% Calligraphic letters (use as needed)
\newcommand{\cF}{\mathcal{F}}
\newcommand{\cG}{\mathcal{G}}
\newcommand{\cB}{\mathcal{B}}

% Notation helpers
\newcommand{\eps}{\varepsilon}
\newcommand{\dd}{\,\mathrm{d}}
\newcommand{\given}{\,\middle|\,}
\newcommand{\st}{\,:\,}

% Norms and absolute values
\newcommand{\abs}[1]{\lvert #1\rvert}
\newcommand{\norm}[1]{\lVert #1\rVert}
\newcommand{\ip}[2]{\langle #1,#2\rangle}

% SDE / stochastic calculus
\newcommand{\dW}{\dd W_t}
\newcommand{\dB}{\dd B_t}
\newcommand{\dt}{\dd t}
\newcommand{\Ito}{It\^o}

% ---------- Exercise numbering as x.y (chapter.exercise) ----------
\newcounter{chap}
\newcounter{ex}[chap]
\renewcommand{\theex}{\arabic{chap}.\arabic{ex}}

% Set current chapter number (resets exercise counter)
\newcommand{\setchapter}[1]{%
  \setcounter{chap}{#1}%
  \setcounter{ex}{0}%
}

% Set the next exercise number within the current chapter
% Example: after \setchapter{2}, \setexercise{3} makes the next exercise be 2.3
\newcommand{\setexercise}[1]{%
  \setcounter{ex}{\numexpr#1-1\relax}%
}

% Exercise environment: optional title
\newenvironment{exercise}[1][]%
{%
  \refstepcounter{ex}%
  \section*{Exercise \theex\if\relax\detokenize{#1}\relax\else\ (#1)\fi}%
}%
{}


% Solution environment
\newenvironment{solution}{\begin{proof}[Solution]}{\end{proof}}

% ---------- Title info ----------
\title{Independent study - Stochastic Differential Equations}
\author{Johan J\"onsson}
\date{January 2026}

\begin{document}

\maketitle
\tableofcontents

% =======================
% Example (Chapter 2, Exercise 9)
% =======================
\setchapter{2}

% Force next one to be 2.9
\setexercise{9}
\begin{exercise}
To illustrate that the (finite-dimensional) distributions alone do not give all the information regarding the continuity properties of a process, consider the following example:

Let $(\Omega,\mathcal{F},P)=([0,\infty),\mathcal{B},\mu)$ where $\mathcal{B}$ denotes the Borel $\sigma$-algebra on $[0,\infty)$ and $\mu$ is a probability measure on $[0,\infty)$ with no mass on single points. Define
\[
X_t(\omega)=
\begin{cases}
1 & \text{if } t=\omega,\\
0 & \text{otherwise}
\end{cases}
\]
and
\[
Y_t(\omega)=0 \qquad \text{for all } (t,\omega)\in[0,\infty)\times[0,\infty).
\]
Prove that $\{X_t\}$ and $\{Y_t\}$ have the same distributions and that $X_t$ is a version of $Y_t$. And yet we have that $t\mapsto Y_t(\omega)$ is continuous for all $\omega$, while $t\mapsto X_t(\omega)$ is discontinuous for all $\omega$.

\end{exercise}

\begin{solution}
By definition 2.2.2 we have that $\{X_t\}$ and $\{Y_t\}$ the same finite-dimensional distribution if
\[
\bP(\{ \omega : X_t(\omega)=Y_t(\omega)\})=1, \, \forall t
\]
we then can in fact show this is true for our case so that will be how we show that they equal in distribution.
\[
\bP(\{ \omega : X_t(\omega)=Y_t(\omega)\})=1-\bP(\{ \omega : X_t(\omega)\neq Y_t(\omega)\})
\]
Then we use that
\[
\bP(\{ \omega : X_t(\omega)\neq Y_t(\omega)\})=\bP(\{ \omega : X_t(\omega)\neq 0\})=\bP(\{ \omega : X_t(\omega)= 1\})=\bP(\{ t\})=0
\]
This holds since it had no mass in a single point. This means that for any $t$ we have that $\bP(\{ \omega : X_t(\omega)\neq Y_t(\omega)\})=0$. This gives that
\[
\bP(\{ \omega : X_t(\omega)=Y_t(\omega)\})=1-\bP(\{ \omega : X_t(\omega)\neq Y_t(\omega)\})=1-0=1, \, \forall t
\]
Specifically we proved something stronger which is that $\{X_t\}$ is a version of $\{Y_t\}$. 
\end{solution}

% =======================
% Chapter 5, Exercise 1
% =======================
\setchapter{5}
\setexercise{1}
\begin{exercise}
Verify that the given processes solve the given corresponding stochastic differential equations ($B_t$ denotes $1$-dimensional Brownian motion).

\begin{enumerate}[label=(\roman*)]
\item $X_t = e^{B_t}$ solves
\[
dX_t = \frac{1}{2}X_t\,dt + X_t\,dB_t.
\]

\item $X_t = \dfrac{B_t}{1+t}$, $B_0=0$ solves
\[
dX_t = -\frac{1}{1+t}X_t\,dt + \frac{1}{1+t}\,dB_t; \qquad X_0=0.
\]

\item $X_t = \sin(B_t)$ with $B_0=a \in \bigl(-\frac{\pi}{2},\frac{\pi}{2}\bigr)$ solves
\[
dX_t = -\frac{1}{2}X_t\,dt + \sqrt{1-X_t^2}\,dB_t
\qquad \text{for } t < \inf\{s>0: B_s \notin \bigl[-\tfrac{\pi}{2},\tfrac{\pi}{2}\bigr]\}.
\]

\item $(X_1(t),X_2(t)) = (t, e^{t}B_t)$ solves
\[
\begin{bmatrix}
dX_1\\
dX_2
\end{bmatrix}
=
\begin{bmatrix}
1\\
X_2
\end{bmatrix}dt
+
\begin{bmatrix}
0\\
e^{X_1}
\end{bmatrix}dB_t.
\]

\item $(X_1(t),X_2(t)) = (\cosh(B_t),\sinh(B_t))$ solves
\[
\begin{bmatrix}
dX_1\\
dX_2
\end{bmatrix}
=
\frac{1}{2}
\begin{bmatrix}
X_1\\
X_2
\end{bmatrix}dt
+
\begin{bmatrix}
X_2\\
X_1
\end{bmatrix}dB_t.
\]
\end{enumerate}
\end{exercise}

\begin{solution}

\begin{enumerate}[label=(\roman*)]
\item 

We begin to observe that 
\[
X_t=g(t,B_t)=e^{B_t} \implies \frac{\partial g}{\partial t}(t,x)=0, \, \frac{\partial g}{\partial x}(t,B_t)=e^{B_t}, \, \frac{\partial^2 g}{\partial x^2}(t,B_t)= e^{B_t} 
\]

\[
dX_t=\frac{\partial g}{\partial t}(t,B_t)\,dt+\frac{\partial g}{\partial x}(t,B_t)\,dB_t+\frac{1}{2}\frac{\partial^2 g}{\partial x^2}(t,B_t)\cdot(dB_t)^2=
\]
\[
e^{B_t}dB_t+ \frac{1}{2} e^{B_t}  \cdot(dB_t)^2 = e^{B_t}dB_t+\frac{1}{2} e^{B_t}  dt= \frac{1}{2}X_t\,dt + X_t\,dB_t
\]

\item We begin to observe that
\[
X_t =g(t,B_t)= \dfrac{B_t}{1+t}, \, B_0=0 \implies \frac{\partial g}{\partial t}(t,x)=\dfrac{-B_t}{(1+t)^2}, \, \frac{\partial g}{\partial x}(t,B_t)=\dfrac{1}{1+t}, \, \frac{\partial^2 g}{\partial x^2}(t,B_t)= 0 
\]
\[
dX_t=\frac{\partial g}{\partial t}(t,B_t)\,dt+\frac{\partial g}{\partial x}(t,B_t)\,dB_t+\frac{1}{2}\frac{\partial^2 g}{\partial x^2}(t,B_t)\cdot(dB_t)^2=
\]
\[
\dfrac{-B_t}{(1+t)^2}dt+ \dfrac{1}{1+t}  \cdot dB_t  = -\frac{1}{1+t}X_t\,dt + \frac{1}{1+t}\,dB_t
\]

\item We begin to observe that

\[
X_t =g(t,B_t)= \sin(B_t), \,B_0=a \in \bigl(-\frac{\pi}{2},\frac{\pi}{2}\bigr) \implies 
\]
\[
\frac{\partial g}{\partial t}(t,x)=0, \, \frac{\partial g}{\partial x}(t,B_t)=\cos(B_t), \, \frac{\partial^2 g}{\partial x^2}(t,B_t)= -\sin(B_t) 
\]
\[
dX_t=\frac{\partial g}{\partial t}(t,B_t)\,dt+\frac{\partial g}{\partial x}(t,B_t)\,dB_t+\frac{1}{2}\frac{\partial^2 g}{\partial x^2}(t,B_t)\cdot(dB_t)^2=
\]
\[
\cos(B_t)  \cdot dB_t  -\frac{1}{2}\sin(B_t) \cdot(dB_t)^2=  \sqrt{1-\sin(B_t)^2}\cdot dB_t  -\frac{1}{2}\sin(B_t) dt= -\frac{1}{2}X_t\,dt + \sqrt{1-X_t^2}\,dB_t
\]
Which holds for $t < \inf\{s>0: B_s \notin \bigl[-\tfrac{\pi}{2},\tfrac{\pi}{2}\bigr]\}$.

\item We begin to observe that
\[
(X_1(t),X_2(t)) = (g_1(t,B_t),g_1(t,B_t))=(t, e^{t}B_t) \implies
\]
\[
\frac{\partial g_1}{\partial t}(t,X)=1, \, \frac{\partial g_2}{\partial t}(t,B_t)=e^{t}B_t, \, \frac{\partial g_1}{\partial x}(t,B_t)=0, \frac{\partial g_2}{\partial x}(t,B_t)=e^t,
\]
\[
\frac{\partial g_1}{\partial x^2}(t,B_t) = 0, \, \frac{\partial g_2}{\partial x^2}(t,B_t)=0
\]
Then we get that
\[
dX_1=1\,dt, \, dX_2=e^tB_t\,dt+e^t\,dB_{t}=X_2 \, dt + e^{X_1} d B_t
\]
\item We begin to observe that
\[
(X_1(t),X_2(t)) = (g_1(t,B_t),g_1(t,B_t))=(\cosh(B_t), \sinh(B_t))=\left(\dfrac{e^{B_t}+e^{-B_t}}{2},\,\dfrac{e^{B_t}-e^{-B_t}}{2}\right)
 \implies
\]

\[
\frac{\partial g_1}{\partial t}(t,X)=0, \, \frac{\partial g_2}{\partial t}(t,B_t)=0, \, \frac{\partial g_1}{\partial x}(t,B_t)=\sinh(B_t), \frac{\partial g_2}{\partial x}(t,B_t)=\cosh(B_t)
\]
\[
\frac{\partial g_1}{\partial x^2}(t,B_t) = \cosh(B_t), \, \frac{\partial g_2}{\partial x^2}(t,B_t)=\sinh(B_t)
\]
Then we get that
\[
dX_1=\sinh(B_t)\,dB_{t}+\frac{1}{2}\cosh(B_t)\,dB_{t}\,dB_{t}=\sinh(B_t)\,dB_{t}+\frac{1}{2}\cosh(B_t)\,dt = X_2 \,dB_{t} + \frac{1}{2}X_1\,dt
\]
\[
dX_1=\cosh(B_t)\,dB_{t}+\frac{1}{2}\sinh(B_t)\,dB_{t}\,dB_{t}=\cosh(B_t)\,dB_{t}+\frac{1}{2}\sinh(B_t)\,dt = X_1 \,dB_{t} + \frac{1}{2}X_2\,dt
\]

\end{enumerate}



\end{solution}


\end{document}

